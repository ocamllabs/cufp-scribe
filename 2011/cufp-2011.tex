% This is file JFP2egui.tex
% release v1.02, 27th September 2001
%   (based on JFPguide.tex v1.11 for LaTeX 2.09)
% Copyright (C) 2001 Cambridge University Press

\NeedsTeXFormat{LaTeX2e}

\documentclass{jfp1}
\bibliographystyle{jfp}
\usepackage{url}

%%% Macros for the guide only %%%
\providecommand\AMSLaTeX{AMS\,\LaTeX}
\newcommand\eg{\emph{e.g.}\ }
\newcommand\etc{\emph{etc.}}
\newcommand\bcmdtab{\noindent\bgroup\tabcolsep=0pt%
  \begin{tabular}{@{}p{10pc}@{}p{20pc}@{}}}
\newcommand\ecmdtab{\end{tabular}\egroup}
\newcommand\rch[1]{$\longrightarrow\rlap{$#1$}$\hspace{1em}}
\newcommand\lra{\ensuremath{\quad\longrightarrow\quad}}

\title[Commercial Users of Functional Programming 2011]
      {Commercial Users of Functional Programming Workshop Report}

\author[Anil Madhavapeddy, Yaron Minsky and Marius Eriksen]
       {ANIL MADHAVAPEDDY\\
        Computer Laboratory, University of Cambridge\\ 
        15 JJ Thomson Avenue, Cambridge CB3 0FD, UK\\
 YARON MINSKY\\
        Jane Street Capital\\
        1 New York Plaza, New York NY, USA\\ 
 MARIUS ERIKSEN\\
        Twitter, Inc.\\
        795 Folsom St., Suite 600\\
        San Francisco, CA 94107, USA.
        }

\jdate{November 2011}
\pubyear{2011}
\pagerange{\pageref{firstpage}--\pageref{lastpage}}
\doi{S0956796801004857}

\begin{document}

\label{firstpage}
\maketitle

\section{Overview}

Commercial Users of Functional Programming (CUFP) is a yearly workshop
that is aimed at the community of software developers who use functional
programming in real-world settings.  This scribe report covers the talks
that were delivered at the 2011 workshop, which was held in association
with ICFP in Tokyo.  The goal of the report is to give the reader
a sense of what went on, rather than to reproduce the full details
of the talks.  Videos and slides from all the talks are available online at \url{http://cufp.org}.

\section{Keynote: Pragmatic Haskell}

Lennart Augustsson from Standard Charter gave the keynote address,
relating his longtime use of Haskell and Haskell-derived languages in
commercial settings.  Augustsson's experience with Haskell dates back to
its inception: he authored the first Haskell compiler, {\tt hbc}, which
remains competitive with the Glasgow Haskell Compiler to this
day. Augustsson subsequently developed several Haskell variants that were
tailored for his needs.

The first of these was {\tt pH}, the parallel Haskell
compiler~\cite{Aditya95semanticsof}. {\tt pH} sought to exploit the implicit
parallelism in a Haskell program by combining the Haskell syntax and type
system with the evaluation strategy of the {\tt id} parallel programming
language~\cite{arvindid}.  {\tt pH} introduced several concepts that are still
used in modern systems, such as the {\tt MVar} abstraction.

Augustsson then applied his Haskell experience to the field of hardware
design.  The Bluespec hardware description language is a full compile-time
implementation of Haskell, outputting Verilog for hardware synthesis.  Bluespec
significantly raised the level of abstraction for hardware design and is still
available commercially today~\cite{Nikhil:2008:BLU:1862867.1862868}. Interestingly,
Bluespec's type system directly incorporates numbers and arithmetic.

Augustsson is currently working in the banking industry, providing in-house
technology for traders and quantitative analysts.  {\tt Mu} is Lennart's latest Haskell
dialect designed specifically for this use.  Most traders and quantitative
analysts are chiefly interested in data transformation and do not
necessarily care about effectful operations.  {\tt Mu} provides a simplified
dialect of Haskell catering to these needs.  Interestingly, Mu eschews
Haskell's laziness and is a strict language instead.  A number of pragmatic design
decisions were made: most performance sensitive code reuses C++ from in-house
libraries, and strings in {\tt Mu} are not lists of characters.  Furthermore,
recursion is provided only optionally (and only 6\% of their modules make use
of it).
{\tt Mu} provides a fine-grained I/O monad, allowing for both "I/O" and "O"
(output only).  In order to attain easy interprocess communication, all values
in {\tt Mu} are serializable.

{\tt Mu} is a true Haskell dialect in that code written in {\tt Mu} may be
compiled with a Haskell compiler.  As with its language and libraries, the {\tt
Mu} compiler is a work of pragmatic design.  All compiler transformations are
done assuming it is operating on terminating code, and the compiler uses {\tt
LLVM}~\cite{Lattner:2004:LCF:977395.977673} for its backend.

Standard Chartered's {\tt Mu} codebase is of significant size and lives within
a library for quantitative analysts written in a combination of C++, Haskell and {\tt Mu}
itself.  Language interoperability is key; all parts of the system can easily
be invoked from Excel, C\#, Java or any other component.

Their experience with strict semantics has been positive.  Particularly useful
is the ease of obtaining meaningful stack traces, tracking resource usage,
debugging and exception propagation.  The chief downside of strict semantics,
in their experience, is the increased difficulty of modular composition.

Augustsson noted that Excel has a more expressive effect system than Haskell and
provides an effective front-end to composing {\tt Mu} modules.  He demonstrated
coding {\tt Mu} within Excel live on stage, via an interactive and interpreted
version.

\section{Cryptol: Theorem-Based Derivation of an AES Implementation}

John Launchbury from Galois then described Cryptol, their declarative
specification language for cryptographic
protocols~\cite{Erkok:2009:PES:1481848.1481860}.  Galois is now using Cryptol
to derive a highly efficient implementation of AES targeted at FPGAs.

Cryptol is a first-order functional language with size-type declarations.
Recursion is available via stream equations.  The sequentiality constraints are
due to data dependencies, making it possible to efficiently evaluate this
language on FPGAs. Cryptography is also a natural match for FPGAs, and Cryptol
makes it easier to experiment with several implementations. Cryptol has a {\tt
theorem} keyword to express properties such as the inverse relationship between
encryption and decryption operations. A tool that is similar to
QuickCheck~\cite{Claessen:2000:QLT:351240.351266} is used to generate test
cases to ensure that these theorems hold.

The derivation of the design was preceded by a series of stepwise refinements to
the original specification, all of which were written in Cryptol.  The
theorem-based testing system was used as a way of gaining confidence that each
stage in the refinement was solid.  The end result was a chip that was
competitive with top-notch manually constructed implementations, but that the
designer could have far more confidence in.

\section{Erlang: Large-Scale Discrete Event Simulation}

Olivier Boudeville from EDF described the use of Erlang for building
large-scale simulations.  Sim Diasca (Simulation of Discrete Systems of All
Scales) is a system implemented in Erlang for building large scale discrete
simulations of the kind that are used for simulating Smart Energy Grids and
other large-scale industry projects.

Sim Diasca was---as its name implies---designed to meet extraordinary
scalability requirements, supporting upwards of 50 million parallel instances
of complex models.  It prescribes no fixed topology and uses Erlang's actor
mechanism for communication between nodes.  The main role of the generic engine
is scheduling.  Its scheduler enforces causality, reproducibility of the model
simulation and some forms of ergodicity.

As typical language choices for this domain are C++ and CORBA, the authors
wrote a macro package---dubbed Wooper---to aid development and provide a
more familiar environment by providing object-oriented primitives on top of
Erlang.

Boudeville characterized their choice of Erlang and functional programming as
``massively positive''.  In particular, Erlang's support for distributing
computation coupled with more suitable language constructs made complex
algorithms significantly easier to express.  The team did encounter resistance
with its esoteric language choice.  In particular, hiring developers was
harder.

\section{Erlang: Testing Safety Critical Automotive Components}

The next talk came from Thomas Arts, who is the CTO of Quviq, a company he co-founded with
John Hughes. Quviq is devoted to automated testing of software using
QuickCheck and Erlang.  The talk covered the application of Erlang and
QuickCheck to the testing of automotive components.  The average modern car has
over 64 computers that are networked (for example, ``the brakes might need to know how fast
the car is going!'').  Manufacturers use components from a variety of vendors
and so interoperability testing takes on critical importance.

The industry standard for automotive components is
AUTOSAR.\footnote{\url{http://www.autosar.org/}} Initially, the AUTOSAR
consortium outsourced its testing, and over thirty person-years were spent on
writing manual tests for component interoperability.  The result was disastrous
and unsuitable for use.  The chief reason for this is that AUTOSAR is highly
configurable, with thousands of standardised parameters that are difficult to
express manually.  Furthermore, this means that meaningful tests must be
parametrized on the configurations.

The solution came in the form of Erlang and QuickCheck.  Quiviq's AUTOSAR
testing produced far better test coverage, tailored to the manufacturer's
particular configuration.  Furthermore, the terseness of Erlang and QuickCheck
reduced the code size of the tests by at least an order of magnitude.

\section{OCaml: Mobile HTML5 Development}

We moved on from testing and simulation to the world of functional web
development in Japan. Keigo Imai delivered an entertaining talk about the
consultancy (IT Planning, Inc.) he works for, where 45\% of their annual sales are from
functional programming projects.

Their typical sales story goes like this. Customers specify their choice of
programming language and an impossibly tight deadline.  ITPL then propose using
their existing OCaml codebase to deliver the solution, but within the required
deadline and with much less risk.  ITPL continually educates its customers
about the benefits of rapid development using functional languages.

Imai then demonstrated an example project: a foreign exchange chart application
that works on iPhone and Android, dynamically drawn using an HTML5 canvas.  It
was written in OCaml and translated to Javascript using the OCamlJS
compiler.\footnote{\url{https://github.com/jaked/ocamljs}}  In their
experience, static typing was very helpful for web development.  In particular,
typing of DOM elements avoided many runtime errors, especially in the complex
{\tt <canvas>} tag.  Higher level language features also facilitate the
asynchronous programming style prevalent in web apps.  Their Web SQL database
API is wrapped in a monad and continuation passing style, allowing
for ease of programming without having to create explicit callbacks chains.
OcamlJS allows for inlining Javascript, which was useful when more low-level
control was required.

Performance has been fine with OCamlJS, and no bugs have been reported yet on
its output. To quote Imai, "it is written in our miraculous super OCaml
technique!". An audience member asked if codesize (from the generated
Javascript) is a problem, especially in view of this being a mobile project.
Keigo replied that they haven't had any problems with this.

\section{Scala: Large Scale Internet Services at Twitter}

Twitter is a social network for sharing short text messages. It is growing in
popularity extremely fast, and some backend components have to serve in excess
of $10^6$ queries per second.  Steve Jenson and Wilhelm Bierbaum described how
many of Twitter's infrastructure components are implemented in Scala.  The JVM
is their preferred virtual machine due to its maturity and performance, and Scala
provides a much better type system and functional programming features than
Java.

All Twitter client HTTP requests are answered by a reverse proxy called TFE,
which routes requests using the Finagle distributed RPC
system.\footnote{\url{http://twitter.github.com/finagle/}} Request streams are
passed through filters that transform and apply processing functions on them.
The process is afforded the full facilities of the JVM, including the use of
pre-existing libraries written in Java.

The use of the JVM means that many existing debugging and profiling tools (e.g.
Yourkit, JStat, VisualVM) can be used, but the name mangling and anonymous
functions in Scala occasionally introduce obscure results.  The use of
immutable values result in a lot of pressure on the garbage collector. When
developing high-volume interactive services, it is not uncommon to spend a
significant amount of time understanding and tuning garbage collection.

Twitter has also been hiring many programmers who have never used Scala before.
While many of them are familiar with the concepts underlying functional
programming, the syntax of Scala can be subtle, and is often difficult for beginners
to learn.

\section{F\#: Mobile Applications using WebSharper}

Adam Granicz from IntelliFactory began by pointing out that the market for
mobile applications is massive.  It is projected that in 2011, the phone market
will have grown to 3.7 billion users, with 18\% of them owning smartphones
capable of running applications.  However, smartphone platforms are a
heterogenous bunch. The current major platforms (Android, iPhone, Windows
Mobile) differ in both programming style and language choice.

Ideally, mobile applications could be developed using higher level
abstractions, support compilations to multiple targets, and make use of desktop
and cloud resources when available. The switch to proprietary and divergent
APIs on different devices is a step backwards.

Javascript is becoming the intermediate language that connects these devices
together, and Windows 8 is even promoting it for building native desktop
applications. IntelliFactory built the WebSharper programming framework in F\#
that outputs Javascript that works with all of these diverse devices.

With WebSharper, all of the server and client code is written in F\#, and
compiles to a complete standalone web application. The F\# interfaces make good
use of the language's facilities: Type-safe URLs helps prevent common errors,
and ``sitelets'' and ``formlets'' that are composable abstractions for
fragments of websites.

\section{Haskell: A Real Time Programming Project in Real Time}

Gregory Wright from Alcatel-Lucent Bell Labs described a project that used
Haskell to build the core of a real-time communication system.  It was built by
the GreenTouch consortium, an organization of equipment vendors, service
providers, and research institutes to show a new antenna technology that
reduces the energy used by wireless networks.  The goal of the project was to
demonstrate an algorithm that reduces radio transmission power as the number of
base station antenna elements is increased and is capable of scaling to
antennas with thousands of elements.

An antenna can focus on a user and calibrate the amount of data sent over the
link. As a result, the power levels of an individual handset can be
adjusted with respect to a target rate.  This was initially simulated and then
later run on real hardware as a soft realtime system.

In total, the project lasted 14 real days. As a result of the timescale, it
used very vanilla Haskell (e.g. arrays weren't unboxed), with no tricks to
improve memory behaviour, or strictness annotations. The STM library was used,
as well as the DSP library from Hackage.

Despite this simplicity, the project was a big success. An important factor was
that they always had a working system, with upgrades staged well. Haskell
provided a high degree of safety from ``crashing and burning''.

Furthermore, the nature of the project was iteself very compatible with the
principles of Haskell. The core algorithms implemented were all
``dataflow-like'', using laziness quite effectively.

\section{Erlang/OCaml: Disco, a MapReduce Platform}

Prashanth Mundkur from Nokia Research in Palo Alto talked about the Disco
Project.  When Nokia started evaluating systems for distributed data
processing, the most popular solution was Hadoop. However, this was a ``massive
pile of Java software'' that was difficult to configure and operate. An Erlang
hacker at Nokia---Ville Tuulos---wondered how hard it would be to implement a
simpler alternative.

The first Disco prototype took a few weeks to implement. The core coordination
components are implemented in Erlang making use of its runtime facilities for
distributed computation. Python is the primary language used to write the data
processing scripts, i.e. the ``mappers'' and ``reducers''. OCaml has recently been
added as a more strongly typed data processing alternative language.

The Erlang environment provides quite a few useful tools that made Disco easier
to implement. In particular, a shell to invoke remote procedure calls on hosts
in the cluster and {\tt fprof} to collect profiling information in real-time.
The dynamic typing did result in some hard to find bugs, and the {\tt dialyzer}
static analysis tool is now used extensively on the codebase~\cite{Sagonas:2007:DDE:1273920.1273926}.

Disco is open-source software, and available from \url{http://discoproject.org}.

\section{mzScheme: Functional DSLs for Game Development}

Dan Liebgold from Naughty Dog Software in Santa Monica then came on stage with
the first gaming related talk at CUFP. They produce the popular Uncharted game
series for the Playstation, which is famous for its complex and interactive
scripted scenes.  Dan described modern game development as a major production
effort where, roughly, artists produce data and programmers produce code.

Naughty Dog has a history of using various Lisp dialects to handle the code and
data in a unified way. But when making the jump from the Playstation 2 to the
Playstation 3, they decided that maintaining a custom Lisp-based game
development system was too costly, and instead dedicated their efforts to
rebuilding the tools, engine, and game in C++ and assembly language.

This decision left no scripting system for gameplay and, more importantly, no
system for creating DSLs and the extensive glue data that is typically required
to develop a major video game. There was no off-the-shelf scripting system that
fit the stringent memory requirements in a Playstation 3, and no language that
would allow rapid DSL creation that fit into the existing tool chain.

With a bit of naivety, a penchant for the Scheme language, and a passion for
functional programming techniques, the team dove in and put together a system
to fill in the gaps!  They used mzScheme\footnote{Now known as Racket,
available: at \url{http://racket-lang.org}}, which can compile to fast native
code. Dan reported that the results have been very good, but not without
issues.  In particular, garbage collector performance sometime led to manual
tuning being required, and build environment integration was tricky.  Syntax
transformations and error reporting led to confusion with new programmers too.

On the other hand, the functional nature of the system was a big win, as it
allowed them to flexibly distill game data down to just the right form to embed
into the resource-constrained run-time environment.  The final result is a
system where programmers, artists, animators, and designers are productively
programming directly in an S-expression Scheme-like language.  Dan closed his
talk by wowing the audience with the trailer for the game, which has now been
released and is garnering extremely positive reviews.

\section{OCaml: The Acunu Storage Platform}

Tom Wilkie from Acunu in London presented their Castle storage system to
optimise big data storage and retrieval. Castle consists of two components: a
management service and storage stack.  The management service is written in
OCaml and handles control requests (eg. snapshots, backups, rollbacks,
coordination). The storage stack implements an efficient disk-backed indexed
key-value store. While developing their storage stack, Acunu needed rapid
prototyping in order to explore new techniques and iterate on ideas. Given that
issues in storage systems tend to surface only after a certain amount of data
or request volume, the prototypes also needed to be reasonably performant.

However, although OCaml was effective for implementing the core algorithm, they
spent a lot of time on the support code to serialise data to and from disk.
When the filesystem codebase was moved to Java, performance increased by 6x
times. The audience suggested a few techniques to speed up the OCaml code,
threading library, the Bitstring syntax extension, and increasing the number of
I/O threads used.  Wilkie acknowledged these could help, but that their
availability should be promoted and documented better.

\section{Conclusion}

This year's CUFP workshop covered a broad set of functional languages---Scala,
Haskell, Scheme, OCaml, Erlang and F\# were all well represented, along with a
nascent but growing interest in more formal tools such as Coq and Isabelle.  We
also ran a well-attended 2-day tutorial series on using Scala, OCaml, Haskell and
F\# for problem solving in scientific computing, web programming, parallel
programming and cloud computing. 

The diversity of industries where functional programming is represented is also
encouraging.  Our speakers worked in the financial industry, big data
processing, safety critical systems in automobiles and energy systems, the
real-time and mobile Internet, and for the first time, the gaming industry.

We would like to thank Michael Sperber for organising the CUFP tutorial series,
and Manuel Chakravarty, Zhenjiang Hu, and the whole ICFP/CUFP team for their
assistance in Tokyo. We look forward to continuing the conference series in
Copenhagen next year!

\bibliography{biblio}
\end{document}

% end of JFP2egui.tex
