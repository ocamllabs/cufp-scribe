% BIBLIOGRAPHY: 
% -- many of the citations come without source: 
% Campall
%% Please provide at least a URL. 
%% also try to make the citations internally consistent. 
%% Thanks 

%% \NeedsTeXFormat{LaTeX2e}
\documentclass{jfp1}

\overfullrule3pt

\usepackage[usenames,dvipsnames]{color}
\usepackage{xspace}
\usepackage{url}
\bibliographystyle{jfp}

\newenvironment{ipar}[0]%
 {\begin{list}{}%
 {\setlength{\leftmargin}{1cm}}%
\item[]%
 }
 {\end{list}}

\newcommand\needcite{{\color{red} [cite]}\xspace}
\newcommand{\note}[1]{ \begin{ipar}  {\color{Gray} \textit{Note}: #1} \end{ipar}}


\title{CUFP'13 Scribe's Report}

\author[Marius Eriksen, Michael Sperber and Anil Madhavapeddy]
       {MARIUS ERIKSEN\\
        Twitter, Inc.\\
        1355 Market Street, Suite 900 \\
        San Francisco, CA 94103, USA.\\
        MICHAEL SPERBER\\
         Active Group GmbH, Hornbergstra\ss{}e 49\\
         70794 Filderstadt, Germany\\
         ANIL MADHAVAPEDDY\\
        Computer Laboratory, University of Cambridge\\ 
        15 JJ Thomson Avenue, Cambridge CB3 0FD, UK}

% \jdate{September 2001, update April 2007}
% \pubyear{2001}
% \pagerange{\pageref{firstpage}--\pageref{lastpage}}
% \doi{S0956796801004857}

\begin{document}
\maketitle

%% \tableofcontents

\section{Overview}

The Commercial Users of Functional Programming workshop (CUFP) is an annual
 workshop held in association with the International Conference on
 Functional Programming (ICFP).  The aim of the CUFP workshops is to
 publicize the use of functional programming in commercial ventures. Its
 motto is ``functional programming as a means, not an end.''

This paper summarizes the presentation of the 2013 event, which took place
 in Boston, Massachusetts, continuing the tradition of the {\it Journal\/}
 to report on the presentations on a yearly
 basis~\cite{JFP:9147276,JFP:8514633}. It sketches the essence of each
 presentations. Curious readers may wish to peruse the recorded videos from
 the workshop.\footnote{Available at \url{http://cufp.org/2013/}.}

The 2013 version of CUFP present a record number of talks, a reflection of
 the growing popularity of functional programming. The talks also covered a
 wide variety of topics, ranging from implementing systems for serving
 advertisement in Erlang to medical device automation in Scheme.

\section{Keynote}

Dave Thomas delivered the 2013 keynote, entitled ``21s Century Crusades of
Knights of the Lambda Calculus---Lessons from Past Language Crusades.''  As
former CEO of OTI, the company responsible for the creation of the Eclipse
IDE, talked about his experiences with the ``language wars'' of decades
past, dealing mostly with the introduction of object-oriented programming
into the mainstream as well as the advocacy of logic programming and expert
systems. Thomas sprinkled his talk with anecdotes and history lessons,
mixing personal opinions with history-based advice to the the functional
programming community.

%% notes.marius.txt:1

Thomas joked that his qualifications included his proclivity to ``infect''
organizations with new technology which they had no intention of actually
using. The only reason he was able to do so was that ``it worked.'' He was
careful to introduce new technologies only when they helped solve actual
problems and when he could find a team that would make full, productive use
of the newly introduced technology.

Next, Thomas suggested that functional programming was experiencing a surge
in popularity because of multicore computing and ``big data''
applications. He warned, though, that the goal of a community should be to
experience \textit{modest} success as massive industrial successes usually
creates a ``really ugly winner,'' attendant with lots of ``really ugly
code.'' Thomas proceeded to deliver a laundry list of important
considerations for a language to succeed commercially.

\paragraph{Language interoperability} A new language must be able to make
use of prior work. Most companies own a large legacy code basis. New
technologies cannot ignore this code base and operate in isolation.  

For example, Tektronix\footnote{See \url{http://www.tek.com/}} was the
first company to commercial deploy Smalltalk for its oscilloscopes. The
project integrated technologies from several ``cultures,'' including
hardware, firmware, systems software, and application software. It
succeeded only because the Smalltalk implementation was {\em designed to\/}
interoperate in this manner.

\paragraph{Have an end-to-end story} A software engineer must be able to
diagnose a system from one end to the other. This aspect is especially
important when solving ``space and time problems.'' Sophisticated JIT
run-times, such as Oracle's HotSpot, are antipatterns in this context. They
make in nearly impossible to understand the run-time behavior of a system.

\paragraph{Serialization; data and code portability} Serialization has
taken a prominent role in the modern computing environment. The
increasingly distributed nature of most systems necessitates good
serialization.  Most of these problems have been solved, and thus new
systems should use prior art. Modern systems need tools that can extract
data from the computing environment; new languages must have fine-grained
control of where this data lives. Distribution also demands easy
\textit{deployment}; a simple way to create deployable executables is a
requirement for a modern language.

\paragraph{Performance} ``Computers like rectangles.''  Language designers
must put care and thought into the handling of arrays and their
implementation. Doing solves many problems and avoids the need for fancy
optimization. Modern architectures rely on caching for performance, thus
control over data locality is paramount as critical parts of your
application must be kept in cache. 

``Stacks are easy to put together but hard to make fast.''  Instead,
implementors should use a register model for a virtual machine. Virtual
machines also need intrinsics to take advantage of modern hardware
architectures.  Intrinsics help future-proof virtual machines because
hardware changes rapidly.

\paragraph{Adoption} Thomas's last point concerns perception. His claim is
that functional languages can \textit{make users feel stupid}. While
dealing with the already-daunting task of understanding foreign ways of
solving problems, newcomers must understand comprehensions, folds, and
monads, which alienates them.  More generally, functional
languages tend to erect high barriers to entry. 

Conversely, if the functional programming community wishes to see its
programming languages adopted, it needs to pay special attention to this
problem by focusing on affordances and education. It needs to empathize
with users and understand that they change at different rates. A recent
example is Haskell's mode for running \textit{wrong} programs. Thomas
considers this step a healthy development.

Finally, much of a typical functional language gets in the way of the
development of ``CRUD (create-read-update-delete) apps,'' which is the kind
of application most of the world's developers are busy writing. Thomas's
point is that functional programming languages ought to be smaller and more
focused than they are now. ``What we need is a collection of little right
languages instead of a smaller collection of partly right and partly wrong
kitchen-sink languages.''

\section{Analyzing PHP Statically}

%% notes.anil.txt:239

Julien Verlaguet (Facebook) presented Hack~\cite{Verlaguet:2014:Hack}, a
statically typed dialect of PHP.

Hack runs on Facebook's PHP virtual machine, HHVM; its compiler is
implemented in OCaml. Much of the HHVM team's efforts are focused on
this ``developer experience,'' with a special emphasis on a rapid
feedback cycle. In particular, the goal is to waste no time between
saving a source file and having the results show up on the screen.

Due to the large size of Facebook's deployed software, performance is
critical.  Even a small performance improvement, say improving CPU
utilization by 1\%, can have a significant cost impact. PHP has been
difficult to optimize due to its dynamic nature, requiring an
ever-more sophisticated virtual machine.

To address this gap between PHP and feasible optimizations, Hack is
statically typed yet compatible with PHP. Indeed, Hack code can
interoperate with legacy PHP code without imposing any run-time
penalty. Thus, developers can adopt Hack on an incremental
basis, similar to gradually typed language though without their
safety~\cite{st:gradual06,thf:dls06}. Hack also comes with type
inference, minimizing notational overhead, and further aiding
adoption.

Hack's type system is interesting in its own right. It handles a large
number of idioms from the dynamically-typed world. At the same time,
it provides sufficiently strong static guarantees to catch a
significant number of errors during type checking.

Facebook's Hack comes with a web-based integrated development environment
(IDE). In order to provide auto-completion and code-navigation features in
the IDE, Hack's type checker is compiled to Javascript via
\texttt{js\_of\_ocaml}~\cite{DBLP:journals/spe/VouillonB14} and runs
alongside the IDE in the web browser.

In keeping with the stated goals, the response time from the compiler is
nearly instantaneous. The type checker processes even thousands of files
within a second, which makes for an impressive live demo.  The Hack
compiler uses a number of resident background processes. A master process
delegates work to a number of child processes, which communicate via
shared memory in a lock-free fashion. This architecture allows Hack to
eagerly type check files before a user could even type the requisite
commands. This is particularly important for large changes, for example,
when a developer switches git branches.

A vast amount of Hack's code is in OCaml. The latter is ideal for symbolic
computation, has excellent performance and can be compiled to Javascript.
The main challenge with the choice of OCaml is the lack of native
multicore support. To address this gap, the Facebook team engineered its
own multiprocess architecture.

%% -----------------------------------------------------------------------------
\section{OpenX and Erlang Ads}

%% notes.anil.txt:389

Anthony Molinaro (OpenX) shared his perspective on his employer's ad
exchange technology.  The advertising technology company developed its
original software in PHP, but it transitioned to Erlang.  As the company's
platform grew, the weaknesses of its software became quite apparent. In
addition to architectural issues---a poor choice of databases, the lack of
HTTP load balancing---the application run-time grew to be an expensive
aspect of OpenX's day-to-day operations. Additionally, OpenX wanted
improved support for concurrent and low latency operations.

Molinaro decided that Erlang was a good fit for the problem space of
highly concurrent systems with soft real-time requirements. He prototyped
a first implementation within a few months. This experience left him quite
satisfied and he started to evangelize it to his coworkers.  In response,
the engineering team began to find additional projects that were suitable
for Erlang:
\begin{itemize}

\item OpenX moved from the Cassandra data-management stack to the
Erlang-based Riak.

\item The developers added Erlang-based services for various elements of
their stack.

\item They also created a DSL for selecting ads so that the application
logic could be interpreted by both Erlang and Java-based systems.

\item OpenX implemented a data-service layer, abstracting the database.

\item Erlang Solutions wrote an API router for OpenX.
\end{itemize}
%
OpenX currently has around 15 services written in Erlang, around 8 in
Java, and a mix of Python and PHP for front-end tasks.

Molinaro emphasized the architectural choices that enabled the
introduction of Erlang:

\begin{itemize}

\item OpenX's cloud-based systems use generic hardware, automate
bootstrapping, are package-oriented and fault-tolerant.

\item OpenX's software infrastructure use many cross-language
tools. Thrift~\cite{Slee:Thrift}, Protocol
Buffers~\cite{Google:2014:Protobufs}, and the Light Weight Event System\footnote{\url{http://www.lwes.org/}} all contribute to a
language-agnostic environment. An in-house system called ``Framework''
provides scaffolding or code layouts and provides support for building
deployable packages from code. Framework also enforces versioning and
reproducibility across languages.

\item OpenX's architecture is service-oriented. Every component has a
single purpose; overall, the components are loosely coupled.

\end{itemize}
While architectural choices enabled the use of Erlang, Malinaro reiterated
that it was important to find a project with which to showcase the
technology.

%% -----------------------------------------------------------------------------
\section{Redesigning the Computer for Security}

%% notes.anil.txt:510

Tom Hawkins (BAE Systems) spoke to us about the Darpa-funded SAFE
project,\footnote{\url{http://crash-safe.org/}}. The SAFE project aimed to
co-design 
\begin{itemize}
\item a new application language, Breeze;
\item a new systems programming language, Tempest;
\item a new operating system; and
\item a new processor.
\end{itemize}
The goal of SAFE was ``security at every level,'' for defense in depth.
SAFE focused on hardware-enforced security, considering dynamic checking
in software as too expensive. 

The SAFE model requires fine-grained information flow control, implemented
in hardware. Words of data in SAFE are called \textit{atoms} comprising
64~bits of meta-data and a 64-bit payload. Atom meta-data contains type
and label information, used by the hardware to perform access and
integrity checks. Every bit of data in SAFE belongs to an atom; meta-data
can always be recovered.

The assembly language is quite traditional, with a few high-level
constructs to make programming the SAFE architecture convenient. It is
implemented as an EDSL in Haskell, using the host language as a macro
language. A monad captures the program description.

Tempest is an imperative language with automatic register allocation, and
an optimizing compiler. It, too, uses the SAFE EDSL assembler as a
backend. Tempest is itself an EDSL in Haskell. This arrangement makes it
straightforward to in-line SAFE assembler.

Hawkins concluded with a handful of lessons, with the humorous aside that
the optimal number of PL researchers on any given project is somewhere
between two and seven:
\begin{itemize}

\item Designing a higher-order language with information flow control
is hard. 

\item Starting with the systems programming language is better than with
an assembler.  While a valuable building block, programming directly in
assembly language is unproductive. Furthermore, having a higher-level
language isolates the software and hardware teams. The hardware team can
change the instruction-set architecture without rewriting all the
software, and the programming language team does not have to anticipate
all hardware features. 

\item EDSLs are excellent for bootstrapping language-intensive systems.
Most importantly, they are also highly reusable components.

\item EDSLs demand that engineers are comfortable with the host language.
They are also more difficult to debug than programs in an external DSL. 

\item Concrete syntax remains important. It is probably best to switch
from abstract syntax to concrete syntax when a language gains modularity,
because then the switch can be made without disruption.
\end{itemize}

\section{End to End Reactive Programming}

%% notes.anil.txt:663

Jafar Hussain of Netflix discussed the company's use of functional
programming techniques, in particular, reactive programming. Netflix
enabled its move toward reactive programming when its development team
modularized the company's software stack. The middle tier and UI used to
be highly coupled, causing numerous problems, especially inefficient calls
between components.

Netflix developers roughly fell into two groups: ``Cloud people'' and
``UI people.'' Due of its massive scale, the company decided that all
developers had to ``think at scale.'' Hence any switch posed the challenge
of how to get UI engineers to think at scale. Netflix answered the
question by providing ``UI comforts,'' a reactive API. 

The next challenge was exploiting parallelism.  Hussain stated that this
parallelism had to come easily, because no developer could be trusted with
locks. He went on to describe the \texttt{Observable} monad, the central
data structure used to compose Netflix's sub-systems. 

{\tt Observable} is part of Netflix's Java port of
Rx.NET~\cite{Christensen:2013:Reactive}. Briefly, it is is a vector
version of the {\tt Continuation} monad with the null-propagation
semantics of the {\tt Maybe} monad and the error propagation semantics of
the {\tt Either} monad. It is composed in a functional fashion, and has
clean cancellation semantics. \texttt{Observable} can be used in either
synchronous or asynchronous settings.

At Netflix, \texttt{Observable} is used as the singular data structure for
both cloud and UI developers. It is applied with ease to problems in
either setting, and it provides a uniform API around which applications
are structured. The video of Hussain's talk includes two demonstrations of
how the API smooths over the differences between cloud and UI programming,
with a uniform structure for both. The first example shows an
implementation of social notifications, a classic ``long polling''
example. The second example is an autocomplete system for search. The
interested reader is referred to the video for details. 

Like other speakers, Hussain emphasized the need for evangelism.  He
recommended practicing public speaking and a particular focus on ``soft
skills.'' In his own experience, he honed his speaking skills in a short
time, and he was then able to clearly articulate the benefits and
importance of a particular technology to his target audience.  Hussain's
group also developed interactive training exercises that illustrated the
benefits of reactive programming. These exercises helped colleagues
develop an intuition of how to use reactive constructs. During the
process, the group's members were almost always available to help.

Netflix has turned its Java and Javascript libraries for reactive
programming into open-source
projects.\footnote{\url{https://github.com/Netflix}}

%% -----------------------------------------------------------------------------
\section{Medical Device Automation using Message-Passing Concurrency in Scheme}

%% notes.anil.txt:817

Vishesh Panchal (Beckman Coulter, Inc.) gave a talk on his team's use of
Scheme to automate a molecular diagnostic device. His talk included a
video of such a device playing a piece of music by actuating its motors.

In general, the purpose of a molecular diagnostic device is to detect the
presence of specific strands of DNA/RNA in a sample. It is a complex
machine, complete with temperature and motor control, sensors, barcode
readers and a spectrometer, plus a total of 19 mother boards. Operators
use a thin, stateless client to interface with the device.  

The server software is written in Scheme. It implements Erlang's
message-passing concurrency model. This library can create, inspect, and
update Erlang-style records. It also provides Erlang's supervisor
structure to isolate hardware failures. Based on the library, the
instrument server is decomposed into several processes, all arranged in a
supervisor tree. These processes communicate by passing messages. The
server also has an event manager that logs events and handles
subscriptions to event streams. The use of message passing renders several
common run-time errors impossible; others are mitigated by the use of
supervisor trees, enabling principled handling of such failures at every
layer in the software stack.

Scientists program the device with a EDSL hosted in Scheme. The language's
implementation greatly benefits from Scheme's hygenic macros, first-class
functions, and continuations. In addition, Scheme's arbitrary precision
arithmetic was important for the numerous numeric computations needed.

The DSL greatly increases the flexible use of the diagnostic devices.
Since scientists are the ultimate users of this DSL, not programmers, the
language is highly specialized; it provides constructs for specifying
high-level goals that are familiar to scientists. The video presentation
includes a number of examples that demonstrate the direct translation of a
scientific process into programs in this DSL. 

Panchal conclusion consists of the following set of lessons: 
\begin{itemize}

\item Message passing is a useful model for reasoning about the semantics
of systems. Erlang/OTP's fault isolation leads to concise programs,
employing only a small amount of defensive code. Concision often implies
ease of testing. It is not a silver bullet, however, especially when
dealing with concurrency and concurrency bugs. For these concerns, it
remains crucial to use timeouts and supervisor trees to detect problems. 

\item Supervisor trees come with little ``prior art.'' Developers are on
their own.

\item Automated unit testing is crucial throughout the process.

\item Hiring into this unique environment remains difficult. 

\item Existing quality metrics (e.g. bug density) do not carry over to
languages such as Scheme due to the terseness of the programs. 

\item Gluing together components written in different DSLs allows
programmers to mix the good bits from both Scheme and Erlang at will. 

\item DSLs also enable rapid prototyping by non-expert programmers. 
\end{itemize}
Finally it is noteworthy that the software also passes the FDA's
scrutiny.\footnote{The US Food and Drug Administration (FDA) is the regulating body for medical devices in
the United States.}

%% -----------------------------------------------------------------------------
\section{Enabling Micro-service Architectures with Scala}

%% notes.anil.txt:936

Kevin Scaldeferri (Gilt Groupe) reported on the experience of building a
large system from a large number of small services.

The Gilt Groupe is an Internet clothing retailer employing highly
user-specific targeting. The company employs several schemes to drive
sales, mostly centered around time- and quantity-limited offerings, which
renders their web traffic rather uneven, with massive spikes around sales
periods. By implication, their revenue is distributed along the same
spikes and valleys, meaning stability during traffic spikes is an
imperative.

The software system at Gilt used to be a largely monolithic Ruby-on-Rails
application.  Scaldeferri explained that, with a growing application, and
a growing number of engineers, there were seemingly intractable software
engineering challenges with this model. The setup also caused a number of
production issues.

The team decided to switch to a ``micro-service'' architecture, splitting
their application into a large number of welf-defined, self-contained
services. The transition began by factoring the Rails application into a
few core infrastructure systems, using HTTP to communicate with each
other. Within four years, these services numbered 300. Each micro-service
was converted to Scala. 

Scaldeferri outlined uses of ``reactive'' programming in this context,
with many examples focusing on real-time updating.  Such updates were
constructed using Play~\cite{Typesafe:2014:Play} and
Akka~\cite{Typesafe:2014:Akka} actors.

During this transition, the engineering group developed several
architectural components to support this large number of services in
production:
\begin{description}
\item[Builds] The group constructed plug-ins for Scala's Simple Build
Tool (SBT) to abstract over build, configuration, and dependency
management.

\item[Configuration] A ZooKeeper cluster stores configurations, which can
be overridden locally. Configurations are deserialized into to Scala data
structures with strict validation.

\item[Testing] Due to the complex set of dependencies testing remains
challenging in micro-service architectures.  Gilt's code base now uses the
``cake pattern'' extensively in testing to fully or partially satisfy
dependencies that would otherwise be handled by another service in their
production environment. The group uses Scala's traits, also known as
mixins, to implement the cake pattern.
 
\item[Delivery] A key part of Gilt's deployment strategy is to deliver
systems on a continuous basis. Twenty to thirty services are deployed
automatically on a typical day.
\end{description}

%% -----------------------------------------------------------------------------
\section{Functional Infrastructures}

Antoni Batchelli (PalletOps) described the Pallet platform for the
automated generation of infrastructure
software.\footnote{\url{http://palletops.com/}}

Pallet allows users to write programs that build and operate computing
environments, both locally and in the cloud. It provides abstractions to
write \textit{plans} that describe configuration actions independently of
the target platform.  It then translates those plans to shell scripts.
While a shell script generated from a plan is specific to a particular
target platform, the plan itself is target-independent.  Pallet currently
knows about several flavors of Linux and Unix systems.

The central entity in a Pallet configuration is a \textit{plan function},
which is a pure function that generates a plan object.  The plan object
can be inspected, that is, the user may query the plan in various
ways. For example, the user can find out what actions a plan would trigger
on a particular platform.

Pallet also optimizes plans.  For example, it can coalesce many small
actions into a single large one if the target operating systems supports
this optimization.  The user can assemble plans into phases that run on
servers. Plans can be abstracted over servers to instantiate entire
families of similar installations.

The implementation of Pallet uses Clojure and exploits Clojure's
multi-methods to specialize actions in plans. While the dynamic typing of
Clojure is an enabling factor for many parts of Pallet, developers
sometimes wish for static typing.

Batchelli concluded by noting that, over time, Pallet put a growing
emphasis on data instead of functions to allow the inspection and
manipulation of plan objects.

%% notes.anil.txt:1028

%% -----------------------------------------------------------------------------
\section{Realtime Map/Reduce at Twitter}

%% notes.anil.txt:1041

Sam Ritchie (Twitter, Inc.) described Summingbird, a new open-source
system for computing aggregates in real time.

Summingbird is a declarative, Scala-hosted EDSL for expressing
map/reduce-style aggregates over streaming data. It bridges the gap
between streaming and batch computation, enabling developers to write
logic once and deploy it in a combination of batch- and
streaming-computation systems. An important goal of Summingbird is to
improve developer productivity by solving the systems problems in one
place, so that the run-time handles efficient execution as well as scaling
resources usage up and down based on need.

Summingbird's core operation is an ``associative plus'' operation, the
Monoid. Its underlying data structure is practical for aggregation. The
associativity of Monoids makes computations parallelizable in a
straightforward way. According to Ritchie, many common data structures and
aggregations that are Monoids, including sets, lists, maps, hyper-log logs,
Bloom filters, moments, count-min sketch, and more. 

It is common to deploy Summingbird in a dual batch/real-time
configuration. The batch portion, working off of log files or ground truth
data, computes the aggregate up to a given time stamp; a real-time
streaming system maintains a sliding window of the same aggregate.
Clients query both of these stores, merging the results.  This style of
deployment is desirable because it cleanly separates two concerns: the
batch system aggregates over the entire data set, optimizing for
throughput and the streaming system has a much smaller fixed window of
computation, computing updates with lower latency.

Ritchie's central example concerns tweets. When Twitter displays a tweet,
it shows a list of web sites that embed this tweet, an act that is based
on impression data. The list is ordered by popularity.  The Summingbird
implementation consumes events of the form \texttt{(TweetId, (URL,
Count))}. The event denotes three facts: the tweet \texttt{TweetId} is
reachable \texttt{Count} times. This fits neatly into the model of
Summingbird, as these tuples are trivially summable. To deal with the
large number of web sites, Twitter uses a Count-Min sketch to reduce
the memory requirements for keeping the counts.

%% -----------------------------------------------------------------------------
\section{Functional Probabilistic Programming}

%% notes.anil.txt:1168

Avi Pfeffer (Charles River Analytics) introduced the \textit{Figaro} language
for probabilistic programming~\cite{Pfeffer:2009:Figaro}. 

The aim of functional probabilistic programming is to ``democratize''
building probabilistic models. A motivating example imagines that someone
has some information and wants to derive answers from this information,
keeping track of the uncertainty of the answers. The solution is to create
a joint probability distribution over the variables, assert the evidence,
and probabilistically infer the answer.

A common approach to probabilistic functional programming uses generative
models. Variables are generated in order such that later values may bind
(depend on) previous values. Developing such a model is not a simple task
and is an active area of research.

Expressions in functional probabilistic programming languages are
computations that produce values with uncertainty. Consider the following
example from Pfeffer's talk: 
\begin{verbatim}
  let student = true in
  let programmer = student in
  let pizza = student && programmer in 
  (student, programmer, pizza)

  let student = flip 0.7 in
  let programmer = if student flip(0.2) else flip(0.1) in
  let pizza = 
    if student && programmer 
      flip(0.9) 
    else
      flip(0.3) in
  (student, programmer, pizza)
\end{verbatim}
Such programs are best understood via a \textit{sampling semantics}. That
is, the program is run many times. Each outcome has some probability of
being generated, the program thus defines a probability distribution over
outcomes. Since the language itself is Turing complete, it is capable of
expressing a wide range of models.

Figaro's central data type is called \texttt{Element[T]}, i.e., the
class of probabilistic models over type \texttt{T}. \texttt{Element[T]}s may be
stochastic or non-stochastic. A number of atomic elements are defined,
e.g. \texttt{Constant}, \texttt{Flip}, \texttt{Uniform}, and
\texttt{Geometric}. These data types are combined to form compound
elements. For example, the compound element, \texttt{If(Flip(0.9,
Uniform(0, 10), Normal(1.0, 0.3))} is the uniform distribution from
$0\mbox{--}10$ $90\%$ of the time, and the normal distribution with mean
$1.0$ and a standard deviation of $0.3$ the remainder of the time.

Figaro uses a probability monad to track state, with \texttt{Constant}
representing the monadic unit, and \texttt{Chain(T, U)} the monadic
bind. Most of Figaro's elements are implemented in terms of this monad.

Figaro is an Scala-hosted EDSL that allows for distributions over any data
type. It has an expressive constraint system, and it comes an extensible
library of inference algorithms containing many popular algorithms. Using
its host language, Figaro can be used as a library with any JVM-based
programming language.  

Figaro is an open-source project~\cite{CRA:2014:Figaro}.

%% -----------------------------------------------------------------------------
\section{Building a commercial development platform Haskell}

%% notes.anil.txt:1290

Gregg Lebovitz (FP Complete) reported on the FP Haskell Center, 
a web-based IDE for Haskell.  

FP Complete aims to improve Haskell adoption and support the Haskell
community in the process. Its primary goal is to make Haskell accessible
to ordinary developers via educational materials and tools. In addition to
free tools, they offer ``commercial grade'' development tools.

The FP Haskell Center greatly simplifies using Haskell. It is a
ready-to-use integrated development environment for Haskell. The IDE
consists of 

\begin{itemize}

\item a web front end;

\item a Haskell back end, implementing project management;

\item integration with the compiler to achieve instant developer feedback,
mostly in the form of error reporting;

\item a help and documentation system;

\item git-based version control;

\item a build system; and

\item an execution and deployment platform.
\end{itemize}

The IDE is itself built almost entirely in Haskell, using libraries and
frameworks, available on Hackage.  The front end uses
Yesod~\cite{Snoyman:2012:Developing} and
Fay,\footnote{\url{https://github.com/faylang/fay/wiki}} a proper subset
of Haskell that compiles to Javascript. The back end continuously
precompiles the user's code so that the bytecode can be run
instantaneously within the IDE.

\section{Common Pitfalls of Functional Programming and How 
to Avoid Them:\\ A Mobile Gaming Platform Case Study}

%% notes.anil.txt:1353

Yasuaki Takebe (GREE, Inc.) spoke about the use of functional programming
in a large, mobile gaming platform (37 million users, mostly in Japan,
2000 games, 2600 employees).  Historically, his company built mobile games
in web programming languages such as PHP or Ruby. Recently, GREE began
using Haskell for some of its back-end systems. Takebe described one of
these projects, a management system for their in-house key-value storage
system.

The systems task is to manage and scale capacity in the company's storage
clusters. It might, for example, increase the cluster size due to hardware
faults or to access spikes.

Takebe presented a few Haskell-specific implementation issues:

\begin{description}

\item[memory leaks due to lazy evaluation] The front-end server kept a
list of active threads in a \texttt{TVar} for monitoring purposes. Operations
to remove threads from this list were evaluated lazily, and could thus
create a large memory leak.

\item[race conditions] A race condition between dequeueing items and an
asynchronous exception thrown by a timeout handler caused the loss of data.

\item[performance degradation] The GREE team used the \texttt{http-conduit}
library to perform health checks of various servers. In a minor version
update, this library started to fork new threads for each http request,
leaving the caller with the responsibility to perform explicit resource
management. As a result, the program ran with as many threads as there had
been health checks, causing a resource leak. 
\end{description}

The GREE team decided to improve their testing practices to battle these
problems. Using standard Haskell tools, especially
QuickCheck~\cite{Claessen:2011:Quickcheck} and
HUnit~\cite{Herington:2014:HUnit}, they developed a harness within which
they could start their servers and test them in-situ. They developed over
150 systems tests with more than 5,000 assertions.

Next they started to document the issues they ran into in order to share
their experiences and prevent future mistakes of the same kind.  While they
put significant effort into this process, few developers bothered reading
them. In response, the team started enforcing a useful subset of the rules
via HLint~\cite{Mitchell:2014:HLint}.

Finally, the group focused on proactive education. They set up a brown-bag
lunch where they covered Haskell and Scala topics. They also ran a class
for new graduates in which students solved problems from project Euler in
Haskell.

Takebe reported that some of GREE's major software component had been
converted to Haskell and considered the project a success. He urged
attendees to pay particular attention to the ``superstructure'' of a
language: its community, the documentation, and the tool chain. Takebe
expressed his belief that these aspects were the sine qua non of
introducing FP in a setting such as GREE.

%% -----------------------------------------------------------------------------
\section{Building scalable, high-availability distributed systems in Haskell}

%% notes.anil.txt:1432

Jeff Epstein (Parallel Scientific) spoke about the use of Haskell in a
high-availability (HA) distributed system for managing resources in a large
(10k+ nodes) cluster manager.  For undisclosed reasons, the team determined
that existing solutions, ZooKeeper among them, were not up to the
task. They therefore set out to build their own implementation of
Paxos~\cite{Lamport:1998:Part} in Haskell.

The job of a cluster manager is to present a consistent view of the
cluster's state and to recover from failures quickly. While the Haskell
implementation employs purely functional data structures employed---the
state of a cluster, for example, is represented by a purely functional
graph---the code itself has an imperative appearance because of the
inherently imperative nature of the domain.

The code uses Cloud Haskell~\cite{Epstein:2011:Towards} for distribution
management. Cloud Haskell is a actor-style message-passing system, similar
to Erlang. It is a particularly good fit for this particular project as its
model of independent, communicating processes meshes well with Paxos.

The Haskell implementation of Paxos is a general purpose library on top of
Cloud Haskell. Each component of the algorithm---the client, acceptor,
proposer, learner, and leader---are Cloud Haskell processes.  Their
implementation consist of about 1.5kLOC of Haskell, closely matching the
pseudo-code in Lamport's original paper.  This kind of near-transliteration
increases the confidence in the implementation's correctness. The team is
now working on adopting modifications from the
literature~\cite{Chandra:2007:Paxos} to improve the library's performance.

As so often, Haskell's lazy evaluation poses problems, chiefly due to space
leaks in low-level networking code. Distribution is a natural barrier to
laziness since messages must be serialized across process boundaries. In
contrast, Haskell's strong typing is a great aid for re-factoring
tasks. Also, since Cloud Haskell provides a platform for distribution, the
language makes it easy to develop and debug distributed systems on a single
machine.  Since such systems generally, and Paxos especially, is sensitive
to non-determinism, it is imperative to use a deterministic scheduler
during development, which allows to test the system in a reliable and
meaningful manner, with the possibility of reproducing errors in a
deterministic fashion. 

%% -----------------------------------------------------------------------------
\section{IQ: Functional Reporting}

%% notes.anil.txt:1525

Edward Kmett (S\&P Capital) discussed the company's use of functional
programming. He started with the introduction of Scala and followed up with
a presentation of \textit{Ermine}~\cite{Compall:2014:Ermine}, a new
Haskell-like language for their domain.

Kmett's team used Scala and FP techniques in a ``portfolio analytics''
engine, a product used for performance and risk attribution across
financial portfolios. The old version of the portfolio analytics engine was
written in Java.  It was hard to extend, and required all data to be in
memory. They rewrote this code in Scala and introduced monoids for simple
parallelization, solving both the performance and extensibility problem.
Reducers were used to derive structure from containers, in a parallel
manner across monoidal structures. Kmett noted how this rewrite removed all
the obvious ``wiring'' from their code.  This project really helped sell
the use of Scala and, more generally, FP to the rest of the company.

Ermine is a Haskell-like language, developed to build a generic reporting
and visualization framework. It is a JVM-based language and will be
used in multiple products.  Like the portfolio analysis engine, Ermine is a
response to problems with a prior implementation---this one in Scala.
Specifically, writing monadic code in Scala can be quite painful as it is
easy to overflow the stack without trampolining.

Ermine has a Haskell-like type system, but with row types, constraint
kinds, and rank-N types. It also provides a built-in database support
sub-system.  Row types are useful in this context as they provide a
powerful mechanism to describe the structure of data. The type system
models constraints with ``has'', ``lacks'' and ``subsumes''.

To support development, Ermine comes with a structured code editor; the
editor prevents programmers from creating code with type errors. In
addition, the language now has a declarative reporting layer that can push
reports into various back ends.

%% -----------------------------------------------------------------------------
\section{Enterprise scheduling with Haskell at skedge.me}

%% notes.anil.txt:1593

Ryan Trinkle (skedge.me) presented skedge.me's cloud-based scheduling
platform.

The company's software handles complex ``enterprise'' scheduling for, among
others, retailers. For example, Sephora, a makeup company, uses skedge.me
to schedule appointments with customers. The skedge.me software is
integrated into Sephora's site via an iframe; it is styled seamlessly to
look like a part of the host site.

Originally, skedge.me consisted of 43,000 lines of code in Groovy on Rails.
The codebase had several major intractable issues: timezone problems,
double bookings, recurring events not firing notifications, and delayed
notifications. The application also had severe performance issues.  Worst
of all, the application was inflexible. It was nearly impossible to respond
to customers' requests; on occasion, they had to ask their customers to
change the way they conducted business to combine a skedge.me model with
the host site. 

After careful deliberation, the team decided to rewrite skedge.me in
Haskell. Trinkle had worked with Haskell previously, though not to
build a web site. The team began by constructing a monad,
\texttt{RawDB}, to maintain ACID (Atomicity, Consistency, Isolation,
Durability) guarantees during transactions.

The revised system consists of three layers. The \texttt{RawDB} layer
tracks effects and can automatically retry operations on temporary failure.
The \texttt{DB} monad is built on top of \texttt{RawDB} monad and provides
a high-level ``CRUD'' interface. This layer makes heavy use of algebraic
data types, and performs caching and validation. The final layer before the
application code is the security layer. It implements security policies for
various customers. Implementing security turns out to be tricky due to the
myriad ways in which the product can be configured by customers. For
example, they may define roles (e.g. ``owner,'' ``staff,'' ``customer'')
that make sense only within their environment; the \textit{verbs} of the
product, too, may be configured (e.g. appointments may be joined or
rescheduled) depending on the environment. Thus both sides of the security
equation---nouns and verbs---can change from instance to instance.  The
skedge.me software uses Haskell's type class facility to model these security
policies. These help map customer-specific customizations into a standard
schema that can be manipulated on a component-by-component basis. This
technique affords the team a great deal of static guarantees from type
checking, a critical property when implementing security sensitive systems.

The team's code base makes heavy use of Hackage, linking in 71 unique
libraries from the repository. An additional 87 are brought in from the
transitive dependency graph. The team considers Hackage to be a well
organized repository. Because most libraries are purely functional in
nature, they are easy to vet for quality. 

Trinkle finally noted that, while Haskell provided a great platform for
``building code for the long run,'' the team also made good use of the
language for quick-and-dirty hacks, e.g., for importing data from the old
system. Even for such hacks, however, Trickle emphasized the strong support
from the type system. Concerning libraries, Trinkle's team replaced just
one library due to bugs. While Haskell might have fewer libraries when
compared to other, more popular languages, those that are available, are of
higher quality.

%% -----------------------------------------------------------------------------
\section{Wolfram: Programming Map/Reduce in Mathematica}

%% notes.anil.txt:1689

Paul-Jean Letourneau (Wolfram) closed this year's CUFP with his talk on
implementing MapReduce in Mathematica. Specifically, Letourneau described
HadoopLink~\cite{Letourneau:2013:HadoopLink}, an integration of Mathematica
and Hadoop.

In Mathematica, everything is an expression. Expressions are rewritten
until the process reaches a fixed point. Expressions are also data
structures, similar to Lisp's S-expressions.  As such Mathematica follows a
familiar LISP mantra of ``programs are data.'' Homoiconicity abounds, which
allows a Mathematica program to manipulate expressions, e.g., to perform
rebinding, which is powerful for distribution.  Letourneau's video presents
some examples of impressively short Mathematica programs. Following Theo
Gray, ``everything is a one-liner in Mathematica~\ldots{} for a
sufficiently long line'', including an image constructed recursively.  In
short, Mathematica as ``a gateway drug to declarative programming.''

HadoopLink allows for nearly seamless distribution of Mathematica programs:
mappers and reducers both are ordinary Mathematica functions, stitched
together by a Hadoop link object for input and output. The programs can be
defined on a single page; HadoopLink takes care of the rest.

Letourneau concluded his talk with the impressive example of a simple
genome search engine. This problem lends itself particularly well to
map/reduce style computation. The program, including large data sets, is
definable within a single Mathematica session. Turning it into a
distributed implementation is indeed a seamless process.

%% -----------------------------------------------------------------------------
\section{Conclusion}

CUFP 2013 was a watershed event. It put an incredible breadth, depth, and
broad applicability of functional programming on display. This was not only
true of the papers presented at the workshop, but also of the many
submissions rejected due to time constraints.

The program was rich in every dimension covered: techniques, languages, and
industries. We had talks from Internet companies, the biotech industry, the
medical device industry, gaming, and the financial industry. Languages in
use varied from EDSLs, to academic stalwarts, to home-grown languages. It
seems that language considerations have taken a front-seat in modern
engineering practices.

As the numerous talks on \textit{service oriented architectures} show, the
process of decomposing a monolithic application into many, smaller services
has been a fertile field for adopting functional programming. Teams use the
transition as an opportunity to re-implement smaller parts of the system in
new languages, which are better suited than the old ones. In other words,
the strategy enables the use of functional programming an incremental and
controlled manner. The advantages of functional programming come across in
a small setting, without having to take large risks.

%% %%%%%%%%%%%%%%%%%%%%%%%%%%%%%%%%%
%% MF: consider dropping this last paragraph
%% %%%%%%%%%%%%%%%%%%%%%%%%%%%%%%%%%
Finally, we would like to thank Simon Thompson and Francesco Cesarini
for organizing this year's tutorials. Ashish Agarwal organized the
evening BoF sessions. We would also like the to thank the ICFP
organizers for their assistance in Boston. We also thank thank
Matthias Felleisen for his editorial help creating this report.

\bibliography{cufp13}

\end{document}
