%% \NeedsTeXFormat{LaTeX2e}
\documentclass{jfp1}

\usepackage[usenames,dvipsnames]{color}
\usepackage{xspace}

\newenvironment{ipar}[0]%
 {\begin{list}{}%
 {\setlength{\leftmargin}{1cm}}%
\item[]%
 }
 {\end{list}}

\newcommand\needcite{{\color{red} [cite]}\xspace}
\newcommand{\note}[1]{ \begin{ipar}  {\color{Gray} \textit{Note}: #1} \end{ipar}}


\title{CUFP'13 Scribe's Report}

 \author[Authors]
        {TODO}


% \jdate{September 2001, update April 2007}
% \pubyear{2001}
% \pagerange{\pageref{firstpage}--\pageref{lastpage}}
% \doi{S0956796801004857}

\begin{document}
\maketitle

\tableofcontents

\section{Overview}

The Commercial Users of Functional Programming workshop (CUFP) is an
annual workshop held in association with the International Conference
on Functional Programming (ICFP), which was held in Boston, MA, USA
Sep 22--24, 2013. Previous workshops have been reported here
\needcite.

The aim of the CUFP workshop is to report on the use of functional
programming in commercial ventures. (To wit, our motto is ``functional
programming as a means, not an end.'')

This report aims to adumbrate; curious readers may wish to peruse
the recorded videos from the workshop \needcite --- CUFP does not
publish proceedings.

\note{Should we expand on the record number of submissions, 
etc? Or on the quality of submissions?}

\section{Keynote: 21s Century Crusades of Knights of the Lambda
Calculus — Lessons from Past Language Crusades}

Dave Thomas, who among much else is responsible for the Eclipse IDE
and the JVM Java Virtual Machine, talked to us about his experiences
of the ``language wars'' in the past; namely, the introduction of
object oriented programming into the mainstream. Dave delivered a talk
sprinkled with anecdotes and history, much opinion, and some advice to
the the functional programming community.

\section{Analyzing PHP Statically}

%% notes.anil.txt:239

Facebook has invested a significant amount of effort in their PHP
infrastructure. Julien Verlauget talked to us about Hack, a statically
typed dialect of PHP used extensively at Facebook. Hack runs on
Facebook's PHP virtual machine, HHVM, and the compiler is implemented
in OCaml.

The sine qua non of software development at Facebook is the rapid
feedback cycle. There should be little time wasted between saving a
source file, and having the results show up on the screen. Much of the
HHVM team's efforts are focused on this, the ``developer experience.''

Due to Facebook's very large deployment, performance is very important.
Even a small performance improvement, say improving CPU utilization 
by 1\%, can have significant cost impact. It is difficult to optimize PHP
because of its dynamic nature, requiring an ever-more sophisticated 
virtual machine.

Hack is compatible with PHP, and indeed interoperates with legacy PHP
with no runtime cost. Due to Facebook's enormous legacy code base, 
Hack is explicitly designed to be incrementally adopted. This is done
by allowing gradual typing. Hack also has type inference, minimizing
notational overhead, and further aiding adoption. Hack's type system
is interesting in its own right: it is flexible enough to handle a large
number of uses where dynamic typing was previously invaluable; at
the same time, it provides sufficiently strong static guarantees to catch
a significant number of errors during type checking.

In keeping with the stated goals, the response time from the compiler
is near instantaneous. Even with thousands of files, the type checker 
returns within a second. Julien demonstrated this to an impressed 
audience.

Facebook has also developed a web-based integrated development
environment (IDE) around Hack. In order to provide autocomplete and
code navigation features in the IDE, Hack's type checker is compiled 
to Javascript via \texttt{js\_of\_ocaml}\needcite and run alongside the 
IDE by the web browser.

The Hack compiler uses a number of resident background processes. A
master process delegates work to a number of child processes; they
communicate via shared memory in a lock-free fashion. This
architecture allows Hack to eagerly type check files before a user
could even type the requisite commands. This is particularly important
when there are large changes, for example when switching git branches.

The vast amount of Hack's code is written in OCaml. The language was
chosen because it is ideal for symbolic computation, has excellent
performance, and can be compiled to Javascript. OCaml's FFI is also
very good, which is important when doing systems work. The main
challenge with this choice is the lack of native multicore support, though
Facebook worked around this by engineering their own multiprocess
architecture.

\section{OpenX and Erlang Ads}

%% notes.anil.txt:389

OpenX is an advertising technology company. Their ad exchange was
written in PHP, but recently transitioned to Erlang. Anthony Molinaro
shared his perspective on this process.

As their platform grew, its squeaky wheels became quite apparent. In
addition to architectural issues --- a poor choice of databases, the
lack of HTTP load balancing --- the application runtime grew to be an
expensive aspect of OpenX's day-to-day operations. Additionally, OpenX
wanted to expand their product, requiring better support for
concurrent and low latency operations.

Molinaro decided that Erlang was a good fit for the problem space ---
highly concurrent systems with soft-realtime requirements --- and
prototyped an initial implementation within a few months. This experience
left him quite satisfied and he started to evangelize it to his coworkers.
The engineering team began to find more and more projects that
were suitable for Erlang:

\begin{enumerate}

\item Moved away from Cassandra to a Riak-based data management
stack. (Riak is written in Erlang.)

\item Added Erlang-based services for various bits and pieces of their
software stack. A DSL for ad-selection was written so that application
logic be interpreted by both Erlang and Java based systems.

\item A data-service layer, abstracting the database.

\item An API router. (Written by Erlang Solutions.)

\end{enumerate}

OpenX currently has around 15 services written in Erlang, around 8 in
Java, and a mix of Python and PHP for frontend tasks.

Molinaro emphasized the importance of the architectural choices that
enabled the use of Erlang:

\begin{itemize}

\item They are entirely cloud based: they use generic hardware,
atuomated bootstrap, deployment via packages, and emphasize fault
tolerance.

\item They use many cross-language tools. Thrift\needcite, Protocol
Buffers\needcite, Lwes\needcite all contribute to a language-agnostic
environment. An in-house system called ``Framework'' provides
scaffolding or code layouts and provides support for building
deployable packages from code. It also enforces versionign and
reproducibility across languages.

\item An emphasis on service oriented architecture. Components are
single-purpose and loosely coupled.

\item An emphasis on evangelism. Architectural choices enabled the use
of Erlang, but it was important to find a project with which to
showcase teh technology.

\end{itemize}

\section{Redesigning the Computer for Security}

%% notes.anil.txt:510

Tom Hawkins spoke to us about the SAFE project, which is funded
by DARPA. SAFE is a codesign of 

\begin{itemize}
\item a new application language, Breeze;
\item a new systems programming language, Tempest;
\item a new operating system; and
\item a new processor 
\end{itemize}

The goal of SAFE is security at every level for defense in depth.
SAFE focuses on hardware enforced security, dynamic checking
in software being too expensive. The SAFE model requires fine-grained
information flow control, implemented in hardware.

Words of data in SAFE are called \textit{atoms} comprising 64 bits of 
metadata and 64 bits of payload. Atom metadata contains type and
label information, used by the hardware to perform access and integrity
checks. Every bit of data in SAFE belongs to an atom; metadata can always
be recovered.

Hardware and software were developed concominantly. While Breeze was
being designed, there was a need for an assembly language to drive the 
hardware, and to write low level routines. The assembly language is quite
traditional, but with a few higher-level constructs to make programming
the SAFE architecture more convenient.

The team implemented an assembly EDSL in Haskell, using the host
language itself as a macro language. The EDSL is very much a typical,
using a monad captures the program description. 

The assembly EDSL was also used as a library to the Breeze compiler;
it worked seamlessly with the compiler's code generator.

However after 2 years, the Breeze language and compiler still isn't in
a usable state, and writing the operating system in assembly language,
even with the help of a macro system, is not a compelling option. The
team concluded there was a need for a higher low-level language. Thus,
Tempest.

Tempest is an imperative language with automatig register allocation,
and an optimizing compiler. It, too, uses the SAFE EDSL assembler
as a backend. Tempest is itself an EDSL in Haskell. This arrangement
made it quite simple to inline SAFE assembler.

Hawkins concluded with a handful of lessons learned:

\begin{itemize}

\item Designing a higher order language with information flow control
is hard. (A humorous aside: the optimal number of PL researchers on
any given project is somewhere between 2--7.)

\item The team would have been more productive by starting by constructing
Tempest initially, and not the SAFE assembler. While a valuable building block,
programming directly in assembly language is unproductive. Furthermore,
having a higher level language isolates the software and hardware teams:
the hardware team can now change the ISA without rewriting all the software!

\item EDSLs are excellent for bootstrapping languages like this; they are also 
highly reusable components, as we have seen.

\item EDSLs do require that engineers are comfortable with the host language
(in this case, Haskell); and they are more difficult to debug.

\item Concrete syntax is still important. The question is when to best make
the switch. Hawkins asserted that the best opportunity to inroduce a concrete
syntax is when a language gains modularity, since then the switch can be made
without disruption.

\end{itemize}

\section{End to End Reactive Programming}

%% notes.anil.txt:663

Jafar Hussain from Netflix talked to us about their use of functional
programming techniques; in particular, reactive programming. Their
path towards reactive programming began by the process of
componentizing the Netflix software stack. The middle tier and UI
used to be highly coupled, leading to among other things very
inefficient call patterns between components since the software
architecture was not modular and forced developers into using
omnibus messages to communicate between the application tiers.

\note{This seems to be a theme: teams decompose large monolithic
applications, and start by tackling niches in which these languages
and techniques are a good fit; then expand from there.}

Developers at Netflix are roughtly dichotomized into two personas:
``Cloud'' people and ``UI'' people. Because of Netflix's rather
massive scale, ``thinking at scale'' must be ubiquitous, and not done
by only the backend infrastructure engineers. Their central challenge
was this: how do they get UI engineers to think at scale? How do they
provide the infrastructure to make them effective ``cloud
developers?''

Netflix answered the question by providing ``UI comforts:'' the use of
Groovy, and a reactive API. The next challenge was then: how do
we mine concurrency and parallelism in a simple fashion? Husain stated
that no developer can be trusted with locks, that there must be better
ways.

Husain went on to describe the \texttt{Observable} monad, the central
datastructure with which Netflix compose their systems. Observable is
a Java port of Rx.NET\needcite. Briefly, it is is a vector version of
the Continuation monad with the null-propagation semantics of the
Maybe monad and error propagation semantics of the Either monad. It
is composed in a functional fashion, and has clean cancellation
semantics. \texttt{Observable} can be used in either synchronous or
asynchronous settings.

At Netflix, \texttt{Observable} is used as the singular data structure
for both cloud and UI developers: it is applied with ease to problems
in either setting, and provides a uniform API around which
applications are structured. Husain demonstrated this by showing
examples of how the API smooths over the differences between cloud
and UI programming --- they really are the same! --- providing a
uniform structure for both. His first example was showing an
implementation of social notifications, a classic ``long polling''
example; the second was an autocomplete system for search.

Like other speakers, Husain emphasized the need for evangelism,
because you simply cannot assume the best technology will win.
He recommended practicing public speaking and a particular focus
on ``soft skills.'' In his own experience, within a short amount of time,
he honed his speaking skills and was able to clearly articulate the
benefits and importance of a particular technology to his target 
audience.

In addition to traditional forms of evangelism, Husain's group
developed interactive training exercises to illustrate the benefits
of reactive programming. These exercises also helped developers gain
an intuition of how to use these new constructs. The group  also made
themselves available at (nearly) all hours of the day to any developer
with questions.

Netflix's libraries for reactive programming, both in Java and in 
Javascript are open source and publicly available\needcite.

\section{Medical Device Automation using Message-Passing 
Concurrency in Scheme}

%% notes.anil.txt:817

Vishesh Panchal from Beckman Coulter Inc gave a talk on their use of
Scheme to automate a molecular diagnostic device. Its job is to
detect the presence of specific strands of DNA/RNA in a sample. It is
a complex beast: it contains a total of 19 boards, with temperature,
motor control, and sensors. It has two barcode readers and a
spectrometer. (Panchal showed us a video of such a device playing a
piece of music by actuating its motors just so!)

Operators use a thin, stateless client to interface with the device.
The server software is written in Scheme. It embeds Erlang/OTP in
order to take advantage of Erlang's message-passing concurrency
model, pattern matching, and supervisor structure, used to isolate
hardware failures.

Scientists program the machine with a DSL, also written in Scheme.
Panchal found this use of Scheme very apposite. Its syntax is
exceptionally clear and simple; hygenic macros played a crucial role
in constructing the DSL; first class functions and continuations were
key; finally, Scheme's arbitrary precision arithmetic was important
for numeric computation.

In order to interoperate with Erlang/OTP, the team constructed an
Erlang embedding in Scheme. Their library can create, inspect, and
update Erlang records; they also replicated the Erlang pattern
matching facility through macros.

The instrument server is decomposed into several processes, all
arranged in a supervisor tree. These all communicate by passing
messages; the server also has an event manager that logs events and
handles subscriptions to event streams.

Panchal enumerated a number of common runtime errors, and showed how
they are impossible in their system, or otherwise mitigated by the
use of supervisor trees, enabling principled handling of such
failures at every layer in their stack.

The flexibility of their diagnostic device is due to its
programmability. However, scientists shouldn't need the help of
software engineers to make use of the device. Thus, their DSL must be
usable by nonexpert scientists. They solved this by making their DSL
highly specialized: it provides constructs for specifying high-level
goals, familiar to scientists. Panchal showed a number of examples
where there was a more-or-less direct translation between a written
description of a process and their DSL.

Panchal concluded by enumerating their chief take-aways:

\begin{itemize}

\item Message passing proved a useful model for reasoning about
system semantics. Erlang/OTP's fault isolation lead to shorter
programs, written less defensively. Smaller programs means simpler
programs, and thus easier to test. However it is not a silver bullet:
In their experience, gen-servers can deadlock between each other.
Here it is crucial to use timeouts and supervisor trees to detect
these.

\item There isn't much prior art on supervisor structure; the team
had to invent new methodology.

\item Automated unit testing was crucial throughout the process.

\item It is difficult to hire into their unique environment.

\item Existing quality metrics (e.g. bug density) doesn't adapt well
to languages like Scheme due to their terseness.

\item There was additional scrutiny from the FDA (the regulating body
for medical devices in the United States) due to their unusual
architecture and use of open source software more generally.

\item Mixing languages, glued together by DSLs and macros, allowed
the team to use the good bits from both Scheme (macros, arbitrary
precision math) and Erlang (message passing concurrency, OTP). The
use of DSLs allowed rapid prototyping by nonexpert developers.


\end{itemize}


\section{Enabling Microservice Architectures with Scala}

%% notes.anil.txt:936

Kevin Scaldeferri from the Gilt Groupe reported on their experience
building out a large system comprising a large number of small
services. The Gilt Groupe is a popular Internet clothing retailer
employing highly user-specific targeting. They employ several schemes
to drive sales, mostly centered around time- and quantity-limited
offerings. This makes their traffic very uneven, with massive spikes
around sales. This also implies that their revenue is distributed
along the same lines, and so stability during traffic spikes is an
infrastructure imperative.

As with Netflix and OpenX, the Gilt Groupe was previously a largely
monolithic architecture, a Ruby-on-Rails application comprised the
bulk of the application. Scaldeferri explained that, with a growing
application, and a growing number of engineers, there were seemingly
intractable software engineering challenges with this model. As well,
the setup caused a number of production issues.

They decided to pursue a ``microservice'' architecture, splitting
their application into a large number of welf-defined, self-contained
services. The transition began by factoring out of the Rails
application a few core infrastructure systems. Within four years,
these numbered 300. Each microservice is written in Scala, and uses
HTTP to communicate with each other. During this transition, the
engineering group developed architectural support to support this
large number of services in production:

\begin{itemize}

\item \textit{Build system}: Scaldeferri's group extended Scala's 
Simple Build Tool (SBT) to abstract away build, configuration, and 
dependency management. They extended the tool (via plugins)
extensively to support their use.

\item \textit{Configuration} is stored in a ZooKeeper cluster, and
can be overriden locally. Configuration is mapped to Scala
type classes with strict validation.

\item Due to the complex set of dependencies --- both upstream and
downstream --- \textit{testing} remains challenging in microservice
architectures. (Scaldeferri remarks: ``It's difficult for developers to run 
300 services on their laptops.'') Gilt's code base uses the ``cake pattern'' --- 
which is enabled by Scala's trait mixin system --- extensively and
uses this in testing to fully or partially satisfy dependencies that would
otherwise be handled by another service in their prodution environment.

\item Finally, a system for doing \textit{continuous delivery} is an
important part of Gilt's deployment strategy. 20--30 services are
deployed automatically on a typical day.

\end{itemize}

Finally, Scaldeferri adumbrated on their use of ``reactive''
programming. Many aspects of Gilt's products require real-time
updating. (Example: they show the number of items left for sale.)
These are constructed using Play\needcite and Akka\needcite actors.

\section{Functional Infrastructures}

%% notes.anil.txt:1028

Antoni Batchelli from PalletOps

\section{Realtime Mapreduce at Twitter}

%% notes.anil.txt:1041

Sam Ritchie of Twitter Inc. described Summingbird, a new open source
system for computing aggregates in real time.

Summingbird is a declarative-style DSL for expressing map/reduce-style
aggregates over streaming data. It aims to bridge the gap between
streaming and batch compuation. You should have to write your logic
once, and deploy it in a combination of batch- and
streaming-computation systems. An important goal of Summingbird is to
improve developer productivity by solving the systems problems in one
place, so that the runtime (i.e. the Summingbird library ands its
underlying execution engines) handles efficient execution as well as
scaling resources usage up and down based on need.

The core datastructure for aggregation in Summingbird is an
``associative plus'' operation --- the Monoid. It is a very practical
datastructure for aggregation: due to their being associative,
computations over Monoids are trivially parallelizable. Ritchie
described a slew of common datastructures and aggregations that are
Monoids. These include Sets, Lists, Maps (whose keys are also
Monoids), HyperLogLog, BloomFilters, Moments, and so on.

It is common to deploy Summingbird in a dual batch/realtime
configuration. The batch portion, working off of log files or ground
truth data, computes the aggregate up to a given timestamp; a realtime
streaming system maintains a sliding window of the same aggregate. At
serving time, clients query both of these stores, merging the result.
This style of deployment is desirable because it allows good
separation of concerns: the batch system aggregates over the entire
data set, optimizing for throughput; the streaming system has a much
smaller fixed window of computation, computing updates with lower
latency.

Ritchie then explained an example production use of Summingbird. When
showing a tweet, Twitter will show a list of web sites that embed this
tweet. This is based on impression data, and is ordered by popularity.
The feature is implemented in Summingbird by consuming events of
the form \texttt{(TweetId, (URL, Count))} representing that
the tweet named by \texttt{TweetId} was reached from a given website,
represented by \texttt{URL}, a given number of times, represented by 
\texttt{Count}. This fits neatly into the model of Summingbird, as
these tuples are trivially summable. In order to deal with the large
cardinality of web sites, they use a \textsc{Count-Min} sketch to 
reduce the memory requirements for keeping the counts.

\section{Functional Probabilistic Programming}

%% notes.anil.txt:1168

\note{This section in particular needs a bit of review; I don't have the
slide material. Need to re-watch the video.}

Avi Pfeffer of Charles River Analytics introduced the Figaro language
for probabilistic programming\needcite. The aim of functional
probabilistic programming is to ``democratize'' buildling
probabilistic models.

Pfeffer started with a motivating example: support you have some
information (e.g. ``Brian ate pizza last night'') and you want to
answer questions based on this (e.g. ``is Brian a student or a
programmer?''), all while keeping track of the uncertainty of the
answers. The solution is to createa joint probability distribution
over the variables, assert the evidence, and then use probabilistic
inference to compute the answer.

A common approach to this is through generative models: variables are
generated in order, wher elater values may bind (depend on) previous
values. Developing such a model is not a simple task, and is still an
active area of research.

Expressions in functional probabilistic programming languages are
computations that produce values with uncertainty. Pfeffer illustrated
by expanding on the previous example:

\begin{verbatim}
let student = true in
let programmer = student in
let pizza = student && programer in
(student, programmer, pizza)
let student = flip 0.7 in
let programmer = if student flip(0.2) else flip(0.1) in
let pizza = if student && programmer flip(0.9) else flip(0.3)
(student, programmer, pizza)
\end{verbatim}

Such programs may be understood by \textit{sampling semantics}:
the program is run many times; each outcome of the program has
some probability of being generated. Thus the program defines a
probability distribution over outcomes. Since the language itself is
Turing complete, it is capable of expressing a wide range of models.

Figaro is an EDSL in Scala which allows for distributions over any
data type. It has an expressive constraint system, and an extensible
library of inference algorithms containing many popular algorithms
such as variable elimination. Due to its embedding in Scala, Figaro
is usable as a library on the JVM, allowing for Java compatibility.
Programs may also make use of the abstractions afforded
by the host language.

Figaro's cheap data type is called \texttt{Element[T]} --- a class of
probabilistic models of type \texttt{T}. A number of atomic elements
are defined (e.g. \texttt{Constant}, \texttt{Flip}, \texttt{Uniform}, and
\texttt{Geometric}); these are combined to form compound elements, 
for example \texttt{If (Flip(0.9, Constant(0.5), ..}.

Figaro uses a probability monad to track state, with \texttt{Constant}
representing the monadic unit, and \texttt{Chain(T, U)} the monadic
bind. Most of Figaro's elements are implemented in terms of this monad.

Figaro is open source and freely available\needcite.

\section{Building a commercial development platform Haskell}

%% notes.anil.txt:1290

Gregg Lebovitz from FP Complete reported on the FP Haskell Center, 
a web-based IDE for Haskell.  FP Complete aims to improve Haskell
adoption by making it more accessible to more developers. They offer
``commercial grade'' tools and support in order to simplify development
in Haskell. They aim to support the Haskell community while doing so.

The FP Haskell Center is aimed to take all the hassle out of starting to
use Haskell. It aims to provide a ready-to-use development environment
for Haskell that is both complete and easy to use. In other words, it is
an Integrated Development Environment (IDE) for Haskell.

The IDE comprises

\begin{itemize}

\item A web frontend, with no setup required;

\item a Haskell backend implementing project management;

\item integration with the compiler to achieve instant developer
feedback (error reporting); 

\item a help help/documentation system;

\item git-based\needcite version control;

\item a build system; and

\item an execution and deployment platform.

\end{itemize}

The IDE is itself built almost entirely in Haskell, using libraries
and frameworks, mostly as-is, available on Hackage. Yesod\needcite 
was used as a web framework. Fay\needcite, a proper subset of Haskell
that compiles to Javascript, was used for frontend code.

The challenging aspects of this project were largely related
to integration. Using the compiler as a library allowed for easy
and type-safe introspection of errors; a user's code is continuously
precompiled on the backend, so that its bytecode can be run
quickly within the IDE.

\section{Common Pitfalls of Functional Programming and How 
to Avoid Them: A Mobile Gaming Platform Case Study}

%% notes.anil.txt:1353

Yasuaki Takebe of GREE, Inc. spoke about the use of functional
programming in mobile gaming. GREE is a large mobile gaming
platform in Japan (37 million users, 2000 games, 
2600 employees) who have historically built mobile games in
traditional web programming languages like PHP or Ruby.

Recently, GREE has begun using Haskell for some of its backend
systems. Takebe described one of these projects, a management system
for their in-house key-value storage system. Its job is to manage and
scale capacity in their storage clusters. It might, for example,
increase the cluster size due to hardware faults or to access spikes.

Takebe described a few incidents that were related to Haskell-specific
issues:

\begin{itemize}

\item \textit{A memory leak due to lazy evaluation.} The frontend server
keeps a list of active threads in a \texttt{TVar} for monitoring purposes.
When threads were removed from the list, the function was not reduced
to normal form and thus created a large memory leak.

\item \textit{A race condition.} \note{NEED DETAILS}

\item \itextit{A performance degradation.} This happened due to forking
too many threads. \note{Need more details}

\end{itemize}

How does the team prevent these issues from occuring again? They
decided to employ better testing practices. Using standard Haskell
tools (QuickCheck\needcite, HUnit\needcite) they developed a 
harness within which they could start their servers and test them 
in-situ. They now have over 150 systems tests with more than 
5000 assertions.

Next, they started to document the issues they ran into, in order to 
share their experiences and prevent future mistakes of the same kind.
While they put a lot of effort into this, few developers bothered reading
them. Next, they started trying to enforce them via HLint\needcite. While
it is not sufficiently expressive to detect all the issues they care about, 
it covers a useful subset.

Finally, the group focused on more proactive education. They set up a
brown-bag lunch where they covered Haskell and Scala topics; they also
ran a class for new graduates where they sourced problems from project
Euler, solving them in Haskell.

Takebe considers their use of FP successful --- some of GREE's key
software is now written in a functional language. He paid particular
attention to the ``superstructure'' of a language: a community, good
documentation, good tooling. Takebe considers this the sine qua non
of introducing FP in a setting like GREE's.

\section{Building scalable, high-availability distributed systems in Haskell}

%% notes.anil.txt:1432

Jeff Epstein of Parallel Scientific spoke about the use of Haskell in
a high-availability (HA) distributed system for managing resources in
a large (10k+ nodes) environment --- a cluster manager. The cluster
manager also needs to be fully consistent, and has to recover from
failures quickly.

The team, for reasons undisclosed, determined that existing solutions,
ZooKeeper among them, were not equal to the task. They set out to
build their own Paxos\needcite implementation with Haskell.

The job of a cluster manager is to present a consistent view of the
cluster's state. While purely functional data structures were employed
--- the state of a cluster, for example, is represented by a purely
functional graph --- the code itself maintained an imperative feel.
This was important due to the inherently imperative nature of the
domain.

The group used Cloud Haskell\needcite for distribution. Cloud Haskell
is a actor-style message-passing system, similar to Erlang. It was a
particularly good fit for this particular project as its model --- of
independent, communicating processes --- meshes well with Paxos.

Paxos was implemented as a general purpose library on top of Cloud
Haskell. Each component of the algorithm --- the client, acceptor,
proposer, learner, and leader --- were all Cloud Haskell processes.
Their implementation turned out to be about 1.5kLOC of Haskell,
closely matching the pseudocode presented in the original Paxos paper.
This kind of near-transliteration provided great confidence in the
implementation's correctness.

While their initial implementation was a simple translation of
Lamport's original Paxos work\needcite, reality is inevitably more
complex. The team are working on adopting modifications availbale
in literator\needcite \note{Multipaxos} to improve performance.

Haskell's lazy evaluation was problematic, chiefly due to space leaks
in low level networking code. Epstein noted that distribution is a
natural barrier to laziness since messages must be serialized across
process boundaries. Epstein cites strong typing as a great aid when
refactoring. Since it provides a platform for distribution, Cloud
Haskell makes it easy to develop and debug distributed systems on a
single machine. This setup greatly aided in development. Since such
systems generally, and Paxos especially, is sensitive to
nondeterminism, the group developed a deterministic scheduler for
Cloud Haskell. This allowed them to reliably and meaningfully test the
system and to reproduce errors.

\section{IQ: Functional Reporting}

%% notes.anil.txt:1525

Edward Kmett from S&P Capital spoke about their use of functional
programming. First, about the introduction of Scala, followed by a
description of \textit{Ermine}\needcite, a new Haskell-like  language they
developed for their domain.

Kmett first used Scala and FP techniques in a ``portfolio analytics''
engine. This is a product used for performance and risk attribution
across financial portfolios.

Here, they used monoids for simple parallelization. (This was also
seen in Twitter's report on Summingbird.). Reducers were used to
derive structure from containers; they are co-monads and compose
rather nicely. \note{What's he talking about here, exactly?} Reduction
is done in parallel across monoidal structures. Kmett noted that this
enhanced compositionality means that ``we don't see the wiring hanging
out of our code any more.''

The old version of the portfolio analytics engine was written in Java.
It was hard to extend, and required all data to be in memory. Their
rewrite in Scala is much more elegant and extensible; it runs fast, in
constant space. This project really helped sell the use of Scala and
FP more generally to the company.

Next, Kmett talked about Ermine, the Haskell-like language they
developed to build a generic reporting and visualization framework to
be used in multiple products.

Ermine was developed because the authors were frustrated with Scala.
Specifically, writing monadic code in Scala can be quite painful as it
is easy to blow the stack without trampolining. Ermine was developed
in both Scala and Haskell; a portable core enables it to run on the
JVM.

Ermine has a Haskell-like type system, but with row types, constraint
kinds, and rank-N types. Ermine has built-in database support. The
group also developed a structured code editor that prevents writing
code that does not type check.

Row types are very useful in this context as they provide a powerful mechanism
to describe the structure of data. The type system models constraints
with ``has'', ``lacks'' and ``subsumes''.

Finally, they built a declarative reporting layer which can push reports
into specific backends (e.g. SQL Server, HTML reports)

\section{Enterprise scheduling with Haskell at skedge.me}

%% notes.anil.txt:1593

Ryan Trinkle skedge.me talked about the cloud-based scheduling
platform skedge.me. Skedge.me handles complex ``enterprise''
scheduling for, among others, retailers. For example, Sephora, a
makeup company, uses skedge.me to schedule appointments with
customers. Skedge.me is integrated into their site via an iframe; it
is styled seamlessly to look at part of the host site.

Skedge.me used to be a 43-KLoC application written in Groovy on Rails.
The codebase had several major intractable issues: timezone problems,
double bookings, recurring events not firing notifications, and
delayed notifications. The application also had severe performance
issues. Perhaps worst of all, the application was inflexible: they
could not respond easily to customers needs, and even had to resort to
asking their customers to change the way they conducted business, so
that it might fit better with the skedge.me model.

After careful deliberation, the team decided to rewrite skedge.me in 
Haskell. Trinkle had worked with Haskell previously, though not
to build a web site.

The team began by constructing a monad, \texttt{RawDB}, to maintain
ACID guarantees during transactions. The \texttt{RawDB} layer tracks
effects and can automatically retry operations on temporary failure.

The \texttt{DB} monad is built on top of \texttt{RawDB} monad and 
provides a high-level ``CRUD'' interface. This layer makes heavy 
use of algebraic data types, and performs caching and validation.

The final layer before the application code is the security layer. It
implements security policies for various customers. Implementing
security turns out to be tricky due to the myriad ways in which the
product can be configured by customers. For example, they may define
roles (e.g. ``owner,'' ``staff,'' ``customer'') that make sense only
within their environment; the \textit{verbs} of the product, too, may
be configured (e.g. appointments may be joined or reschedueld)
depending on the environment. Thus both sides of the security equation
--- nouns and verbs --- can change from instance to instance.
Skedge.me uses Haskell's type class facility to model these security
policies. These help map customer-specific customizations into a
standard schema that can be manipulated on a component-by-component
basis. This technique affords the team a great deal of static
guarantees from type checking, a criticial property when implementing
security sensitive systems.

The team made heavy use of Hackage. Their application brings in 
71 unique libraries from Hackage. (An additional 87 are brought in
from the transitive dependency graph.) They found Hackage to be 
well organized. Because most libraries are purely functional in nature,
the team found them very easy to vet for quality. They replaced just
one library due to bugs. Trinkle made this observation: compared to 
the available libraries of other, more popular languages, Haskell's 
number fewer, but are of higher quality.

Trinkle finally noted that, while Haskell provides a great platform
for ``building code for the long run,'' the team also made good use of
the language for quick-and-dirty hacks. As an example, the program
that imports data from the old system was written in Haskell. It was
not written with modularity in mind --- it was, truly, a
quick-and-dirty-hack --- however, the type system still provides very
useful guarantees. Trickle noted that they couldn't have acheived
doing this ``as badly as they did'' without the aid of the type
system.

\section{Wolfram: Programming Map/Reduce in Mathematica}

%% notes.anil.txt:1689

Paul-Jean Letourneau of Wolfram closed this year's CUFP with his talk
on implementing MapReduce in Mathematica. Letourneau first described
the Mathematica language: everything is an expression; expressions are
transformed (rewritten) until they reach a fix-point. Expressions are
data structures (similar to LISP's m-structures). As such Mathematica
follows a familiar LISP mantra: programs are data --- homoiconicity
abound. This allows a Mathematica program to manipulate expressions
--- e.g. to perform rebinding --- which is powerful for distribution.

Letourneau shared some examples of impressively short Mathematica
programs (``Everything is a one-liner in Mathematica... for a sufficiently
long line'' ---Theo Gray) including an image constructed recursively. 
Letournea described Mathematica as ``a gateway drug to declarative
programming.''

Next, Letourneau described HadoopLink\needcite, a system that integrates
Mathematica with Hadoop. HadoopLink allows for nearly seamless 
distribution of Mathematica programs: mappers and reducers both
are ordinary Mathematica functions, stitched together by a Hadoop
link object for input and output. The programs can be defined on a single
page; HadoopLink takes care of the rest.

Letourneau then went on to build a simple genome search engine using
these primitives (this problem lends itself particularly well to map/reduce
style computation). The program, including referring to large datasets,
was defined within a single Mathematica session; distribution was indeed
seamless.

Letourneau's talk was a fitting end to an excellent CUFP workshop.

\bibliography{cufp13}

\end{document}
