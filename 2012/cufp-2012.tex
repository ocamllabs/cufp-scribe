% This is file JFP2egui.tex
% release v1.02, 27th September 2001
%   (based on JFPguide.tex v1.11 for LaTeX 2.09)
% Copyright (C) 2001 Cambridge University Press

\NeedsTeXFormat{LaTeX2e}

\documentclass{jfp1}
\bibliographystyle{jfp}
\usepackage{url}

%%% Macros for the guide only %%%
\providecommand\AMSLaTeX{AMS\,\LaTeX}
\newcommand\eg{\emph{e.g.}\ }
\newcommand\etc{\emph{etc.}}
\newcommand\bcmdtab{\noindent\bgroup\tabcolsep=0pt%
  \begin{tabular}{@{}p{10pc}@{}p{20pc}@{}}}
\newcommand\ecmdtab{\end{tabular}\egroup}
\newcommand\rch[1]{$\longrightarrow\rlap{$#1$}$\hspace{1em}}
\newcommand\lra{\ensuremath{\quad\longrightarrow\quad}}

\title[Commercial Users of Functional Programming 2012]
      {Commercial Users of Functional Programming Workshop Report}

\author[Michael Sperber and Anil Madhavapeddy]
       {MICHAEL SPERBER\\
         Active Group GmbH, Hornbergstra\ss{}e 49\\
         70794 Filderstadt, Germany\\
         ANIL MADHAVAPEDDY\\
        Computer Laboratory, University of Cambridge\\ 
        15 JJ Thomson Avenue, Cambridge CB3 0FD, UK}

\jdate{May 2012}
\pubyear{2012}
\pagerange{\pageref{firstpage}--\pageref{lastpage}}

\begin{document}

\label{firstpage}
\maketitle

\section{Overview}

Commercial Users of Functional Programming (CUFP) is a yearly workshop
that is aimed at the community of software developers who use functional
programming in real-world settings.  This scribe report covers the talks
that were delivered at the 2012 workshop, which was held in association
with ICFP in Copenhagen.  The goal of the report is to give the reader
a sense of what went on, rather than to reproduce the full details
of the talks.  Videos and slides from all the talks are available online at \url{http://cufp.org}.

\section{Keynote: Adopting Functional Programming}

Kresten Krab Thorup, CTO of Trifork, Aarhus gave the keynote address,
and took us on the voyage he had taken from being on ``object head''
to ``Erlang land.''  Thorup's foundational training in software
development was all in terms of object-oriented methodologies.  He
went on to work on Objective C for NeXT, afterwards took his Ph.D.\
and then founded Trifork, an IT services company with now about 250
employees that develops software solutions, provides training, and
organizes several well-respected conferences.

While Trifork originally capitalized almost exclusively on its Java
expertise, it now successfully applies Erlang in large-scale
industrial projects.  Thorup described (taking cues from anthropology)
how many organizations have not been able to make such transitions
easily: Groups tend to gather around an idea that keeps them together,
and try to keep new ideas at bay.  This makes it difficult for
long-time OO developers to adopt functional programming.

Trifork managed to stay flexible by making learning about new ideas
and communicating them part of their regular operation: Everyone at
Trifork should spend 10\% of their time in the structured exchange of
knowledge, by giving presentations, organizing meetings, give training
classes, or organizing conferences.

Thorup reviewed object-oriented programming and the ecosystem around
it to show how it had become successful: Through an intuitive idea~--
``an object is an independent encapsulated entity that inteprets
inputs on its own account''~-- but also because of the availability of
thinking tools: graphical notation for design, tools for mapping those
designs to programs, books on comon problems akin to ``Design
Patterns''~\cite{GammaHelmJohnsonVlissides1995}, analysis methods for producing systems,
and standardized qualification processes.

However, Thorup sees a series problem with the object-oriented model,
as objects have no coherent model of time, and no good way to compose
behaviors over time.  With the rise of multicore and distributed
computing, these become increasingly important.  Erlang, supporting
functional programming and an actor model for concurrency, parallelism
and distribution, addresses this issue.  Thorup stressed that Erlang
is not primarily a functional programming language, but that
functional programming helps Erlang meet its requirements: being a
language for writing robust distributed applications.

Thorup described large-scale projects done using Erlang: One for
managing health-care records in Denmark, and one for sharing data
among ``sometimes connected devices'' such as a cell phones at a music
festival.

Thorup concluded by noting that two fundamental classes of problems in
software development require different classses of solutions:
\textit{Interactive systems} with multiple parties are fundamentally
stateful, and where developers should understand the handling of
state~-- for those problems, actors are a good model.
\textit{Transformational systems} map input to output, where
developers want to abstract over the details of hardware utilization,
the handling of mutable state and coordination.

\section{Jane Street Status Report}

% A Caml's Progress
% 
% Jane Street is mired in FP, but not in transitional environment; not
% academic, not theorem proving, etc.
% 
% This talk is an update on where they are on using OCaml after more
% than a decade.
% 
% --
% 
% The Jane Street company interacts with lots of people, anonymously,
% over the world's securities markets.  They have 320 employees in three
% continents, with billions of dollars per day traded; much more than
% the value of the company.  In this situation, it is easy to go out of
% business is a mistake is made.
% 
% There is a lot of exogenous complexity in their environment:
% regulation, new security types, many other actors to interact with,
% etc.
% 
% Their software has three main kinds of requirements:
% 
% * Correctness - most important, related to the "easy to go out of
% business" point above
% 
% * Agility - need to quickly exploit new opportunities as they're found
% 
% * Performance - of course
% 
% --
% 
% Application areas where Jane St. has written OCaml code:
% 
% * Research Tools - investigating trading strategies with statistics; prediction
% 
% * Trading Systems - was VB in Excel.  used to make trades automatically
% 
% * Order Gateways - programs that interact with markets, talking
%  others' protocols, having to deal with differences between the
%  specification and reality.
% 
% * Post-Trade - internal understanding of completed trades, cleanup
%  related to closing houses
% 
% * Systems Infrastructure - management of systems; IT
% 
% * Dev Tools - 100 devs (including part time); OCaml is a good language
%  but has bad tools
% 
% * Trading Tools - visualization using curses; prevent bad UX designers
%  from doing UX
% 
% * Market Data - pull data from exchanges around the world
% 
% * Desk Infrastructure - trading desks' tools build (by traders) in
%  OCaml; one month training in OCaml for new traders
% 
% --
% 
% Platform:
% 
% * Core - stdlib replacement
% 
% * Async - sort-of-threads; it is okay to have different abstractions
%  for different things
% 
% * Incremental - large scale computations w/ small updates
% 
% * Catalog - pub/sub system w/ high level architecture abstractions
% 
% * Nile - distributed message passing (in progress)
% 
% --
% 
% Future:
% 
% * More OCaml - committed fully
% 
% * More Open Source - improve community by contributing code back to it
% 
% * More Collaboration - improve core technology, tools
% 
% ==
% 
% Q: What is your experience with the usual pattern of increased
% rigidity/obscurity with large systems?
% 
% A: 1) Can refactor ruthlessly because of strong typing
%   2) avoid clever tricks, as/when they are hard to explain to others
% 
% Q: FP is usually described as 10x more concise than imperative, in
% your experience, is it 9x?  11x?
% 
% A: don't want concision first, want explicitness.  Don't have a real #
% on rewrites from other code to OCaml because of adding features.
% 
% Q: What failures happen often?
% 
% A: Jane St. code fails by turning off; this happens, but it's okay.
% Failure causes are usual problems, load, incorrect
% specification/understanding of world, etc.

\section{Transmitting customised ads to set-top boxes with Erlang}

\section{Haskell for XenClient}

\section{Functional Programming @ Ghent IT Valley}

% Slides here:
% http://www.slideshare.net/toolslive/cufp-2012-talk

\section{From Streams to Functions (and Back Again)}

\section{Functional Big-Data Genomics}

\section{Using F\# to Prove Stabilization of Biological Networks}

\section{Developing an F\# Bioinformatics Application with HTML5
  Visualization}

\section{Developing Medical Software in Scala and Haskell}

\section{Functional programs connected to the power grid}

\section{Clojure iPad analytics dashboard in energy sector}

\section{The Awesome Haskell FPGA Compiler}

\section{Conclusion}

We would like to thank Simon Thompson and Duncan Coutts for helping to
organize the CUFP tutorials, and Ashish Agarwal for organizing the BoF
sessions.  Also, we thank Peter Thiemann and Fritz Henglein and the
whole ICFP/CUFP team for their assistance in Copenhagen.  We look
forward to CUFP 2013 in Boston!

\bibliography{biblio}
\end{document}

% end of JFP2egui.tex
